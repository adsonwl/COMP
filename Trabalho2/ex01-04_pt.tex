%%%%%%%%%%%%%%%%%%%%%%%%%%%%%%%%%%%%%%%%%%%%%%%%%%%
% Exemplo 04: alinhamento do texto
% ultima atualizacao: 14/05/2004 por Sadao Massago
% http:/www.dm.ufscar.br/~sadao
%--------------------------------------------------
\documentclass[12pt,a4paper]{article}


% determina o espa�amento. Por exemplo, 1.5 ser� um espa�o e meio
% \linespread{1.5}

% tomada de decis�o quando ocorre espa�amento ruim (bad box)
% padrao eh \fussy: preferencia a overflow
% \fussy 
% e
% sloppy: preferencia a underfull
\sloppy

\begin{document} % inicio do documento

O espa�amento de linhas no documento � especificado pelo comando
\verb|\linespread{valor}| no \texttt{preamble} (antes do \verb|begin{document}|,
onde \texttt{valor} � o valor num�rico real, em rela��o ao espa�amento normal.
Por exemplo, \verb|\linespread{1.5}| ser� um espa�o e meio e 
 \verb|\linespread{2}| ser� espa�o duplo.
 

\verb|\fussy|

usado como padr�o produz ``overfull'' (n�o cabe no espa�o dispon�vel) quando 
n�o consegue acertar o espa�amento. isto deixa o texto invadir a margem, 
sendo necess�rio corrigi-los.

\verb|\sloppy|

Produz o ``underfull'' (muito espa�o entre palavras) quando n�o consegue acertar
o espa�amento. O aplicativo de office (como MS Word) usa este modo. 
Apesar de poder imprimir sem corrigi-lo, o texto pode ficar feio devido ao 
espa�o exagerado entre palavras.

O \verb|fussy| e \verb|sloppy| pode ser aplicado em qualquer trecho do documento
e ter� efeito at� aparecer um deles novamente. Para aplicar no documento
inteiro, coloque no \texttt{preamble} (antes de \verb|\begin{document}|).

O alinhamento padr�o � justificado, mas poder� alterar num determinado trecho:

% Para centralizar, coloque entre 
% \begin{center}
% e
% \end{center}

\begin{center}
    Esta parte ficar� centralizada.
\end{center}

% para alinhar a esquerda, coloque entre
% \begin{flushleft}
% e
% \end{flushleft}

\begin{flushleft}
Esta parte ficar� alinhado a esquerda.
\end{flushleft}

% Para alinhar a direita, coloque entre
% \begin{flushright}
% e
% \end{flushright}

\begin{flushright}
Esta parte ficar� alinhada a direita
\end{flushright}

% Note que, quando est� dentro do ambiente
% delimitado pelo 
% \begin{nome do ambiente}
% e 
% \end{nome do ambiente}
% entao podemos colocar simplesmente
% \center
% \flushleft
% ou
% \flushright
% sem necessidade de usar o begin/end
% ex.: o ambiente figure define figura
% \begin{figure}
% \center
% aqui ficar� centralizada
% \end{figure}

 
\end{document} % final do documento
