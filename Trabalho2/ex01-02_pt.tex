%%%%%%%%%%%%%%%%%%%%%%%%%%%%%%%%%%%%%%%%%%%%%%%%%%%%%%%%%%%
% Exemplo 02: acentuacao no modo tex e caracteres especiais
% ultima atualizacao: 14/05/2004 por Sadao Massago
% http:/www.dm.ufscar.br/~sadao
%---------------------------------------------------------- 
\documentclass{article}

\begin{document} % inicio do documento
% Acentu��o no Modo tex
% \ seguido de acento e caracter a ser acentuado
% ex.: \`agua, voc\^e, etc. a letra a ser acentuada pode estar entre chaves
% ex.: \`{a}gua, voc\^{e}, etc.
% s�o aceitos:
% \` -> crase
% \' -> acento agudo
% \^ -> acento circunfrexo
% \" -> trema
% \= -> barra
Pode acontecer de precisar editar o c\'odigo fonte num sistema em que n\~ao tem 
fontes acentuadas (para mostrar na tela), ou que o teclado n\~ao aceite a acentua\c c\~ao direta.
Neste caso, dever\'a utilizar a acentua\c c\~ao no modo tex.
Por exemplo, poder� produzir \=a.

% i e j sem pingo eh produzido por \i e \j
Para n\`ao ter problemas de pingo nas letras acentuadas, tem o i e j sem pingos, como estes:
\i, \j

Ex.: \'\i tem ou \'{\i}tem

% Para colocar sedilha, usa-se o \c seguido de espa�o e c, ou seguido de {c}
% ex.: ma\c c\~a, ma\c c\~{a}, ma\c{c}\~a, ma\c{c}\`{a} produzem mesmo efeito.


% apostrofos e aspas:
% apostrofo eh aberto com um ` e aspas, com `` (dois ` seguidos)
% apostrofo eh fechado com ' e aspas com '' (dois ' seguidos)
``Podemos escrever entre `aspas' e `apostrofos' sem problemas''.
% Tome cuidado de n�o usar " para fechar o aspas. Sempre use '',
% apesar de alguns tex antigo trata " como fechar aspas.
% Outro cuidado: Se estiver usando teclado portugues, 
% n�o confunda o acento agudo com apostrofos

% No caso de aspas acima, o latex troca segu�ncia de caracteres dois ` ou dois ' 
% por um novo caracter. Isto � denominado de "ligadura". Outros exemplos de
% ligadura s�o:
% -- que � travess�o, --- que e travessao longa, ?` e !` que s�o ponto de 
% interroga��o e exclama��o de "ponta cabe�a"
% Tamb�m existem outras ligaduras que ele usa e a gente nem nota, 
% tal como trocar "fi" ou "ff" pela caracter "fi" e "ff" que 
% s�o meio grudados 9como pouco espa�amento entre eles) para deixar 
% o texto mais bonito. Este tipo de ligadura 
% interno pode dar prblemas quando usa codificacao de fonte inadequada
% (o que denomina "problema de ligadura em ...").
%%%%%%%%%%%%%%%%%%%%%%%%%%%%%%%%%%%%%%%%%%%%%%%%%%%%%%%%%%%%%%%%%%%%%%%%%%%%

podemos escrever ?` e !` assim como -- e ---.

% Caracteres especiais
% seguints caracteres especiais sao especificados, colocando \ antes dele:
% $, #, %, &, _, {, }
Caracteres especiais podem ser escritos no texto:
\$, \#, \%, \&, \_, \{, \}

% caracteres que usam como acento n�o podem ser seguido esta regra.
% entao o truque � acentuar sobre vazio especificado como {}
% ex.: \`{}, \'{}, \~{}, \^

Acentos sobre vazio: \`{}, \'{}, \~{}, \^{}

Caracteres n\~ao especiais (apesar de acento): " (n\~ao use este para fechar aspas).

% Note que \\ � reservado para linebreak e 
% \ seguido de espaco e reservado para espaco extra.
% Assim, nao pode seguir a regra acima. 
% dever� escrever $\backslash$ ou \textbackslash 

Podemos escrever $\backslash$  ou \textbackslash

% o sinal de simetria pode ser usado como tilde

um tilde que n\~ao fica centrada (na altura): $\sim$

% um espa�o estra inserido pelo \ seguido de espa�o pode ser 
% dividido (metade na linha e resto na outra linha)
% um espa�o inserido pelo ~ n�o pode ser dividido

% veja a p�gina~1 
% 1 n�o pode ser separada de p�gina, apesar de ter espa�os
% cuidado para n�o colocar espa�o junto com ~ como
% p�gina~ 1
% ou
% p�gina ~1
% pois perde o efeito de manter juntas

O espa�o extra que pode ser dividido pode ser inserido com contra barra seguido de espa\c co como abaixo:

um \ \ \ \ \ espa�o \ \ \ \ \ extra \ \ \ \ entre \ \ \ \ palavras.

% Um c�digo pode ser inserido no meio do texto usando \verb
% \verb+[texto]+
% onde + est� indicando o comeco e fim do bloco. Ele pode ser trocado pelo 
% outro par de caracteres que n�o seja *, conforme a necessidade

\
 
Nota: O logotipo \tex, \latex{} e \latexe{} tamb\'em podem ser inseridos no documento, mas lembre-se que
os comandos do \tex{} \'e sens\'\i vel ao mai\'usculo/min\'usculo.

Nota: Em geral, os espa\c cos depois do comando s\~ao ignorados.
Para inserir espa\c cos depois do comando, basta colocar par de chaves ap\'os o comando.
Por exemplo, para produzir ``\latex{} produz...'', 
escreva \verb*+\latex{} produz...+, mas n\~ao utilize ``+\ +'' nem ``$\sim$'' para inserir o espa\c co, 
pois ter\'a espa\c co indesejado quando ele cair no final da linha.

\end{document} % final do documento
