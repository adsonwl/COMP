\documentclass{article}

\title{Documento teste 2}

\begin{document}

%descricao da linguagem LaTex

\section{LaTeX}


\bf{Latex} (normalmente formatado como \bf{LaTeX}) e um conjunto de macros para o programa de diagramacao de textos \textit{TeX}, utilizado amplamente na producao de textos matematicos e cientificos, devido a sua alta qualidade tipografica. Entretanto, tambem e utilizado para producao de cartas pessoais, artigos e livros sobre assuntos diversos.


\section{Utilizacao}

\paragraph{Paragrafo 1:}
A ideia central do \bf{LaTeX} e distanciar o autor o maximo possivel da apresentacao visual da informacao, pois a constante preocupacao com a formatacao desvia o pensamento do conteudo escrito.
Ao inves de trabalhar com ideias visuais, o usuario e encorajado a trabalhar com conceitos mais logicos — e, consequentemente, mais independentes da apresentacao — como capitulos, seções, ênfase e tabelas, sem contudo impedir o usuario da liberdade de indicar, expressamente, declaracoes de formatacao.

\end{document}
